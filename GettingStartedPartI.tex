% Options for packages loaded elsewhere
\PassOptionsToPackage{unicode}{hyperref}
\PassOptionsToPackage{hyphens}{url}
\PassOptionsToPackage{dvipsnames,svgnames*,x11names*}{xcolor}
%
\documentclass[
]{article}
\usepackage{lmodern}
\usepackage{amssymb,amsmath}
\usepackage{ifxetex,ifluatex}
\ifnum 0\ifxetex 1\fi\ifluatex 1\fi=0 % if pdftex
  \usepackage[T1]{fontenc}
  \usepackage[utf8]{inputenc}
  \usepackage{textcomp} % provide euro and other symbols
\else % if luatex or xetex
  \usepackage{unicode-math}
  \defaultfontfeatures{Scale=MatchLowercase}
  \defaultfontfeatures[\rmfamily]{Ligatures=TeX,Scale=1}
\fi
% Use upquote if available, for straight quotes in verbatim environments
\IfFileExists{upquote.sty}{\usepackage{upquote}}{}
\IfFileExists{microtype.sty}{% use microtype if available
  \usepackage[]{microtype}
  \UseMicrotypeSet[protrusion]{basicmath} % disable protrusion for tt fonts
}{}
\makeatletter
\@ifundefined{KOMAClassName}{% if non-KOMA class
  \IfFileExists{parskip.sty}{%
    \usepackage{parskip}
  }{% else
    \setlength{\parindent}{0pt}
    \setlength{\parskip}{6pt plus 2pt minus 1pt}}
}{% if KOMA class
  \KOMAoptions{parskip=half}}
\makeatother
\usepackage{xcolor}
\IfFileExists{xurl.sty}{\usepackage{xurl}}{} % add URL line breaks if available
\IfFileExists{bookmark.sty}{\usepackage{bookmark}}{\usepackage{hyperref}}
\hypersetup{
  pdftitle={Getting Started with R, RStudio, \& R Markdown for ENVH 556: Part I},
  pdfauthor={Lianne Sheppard},
  colorlinks=true,
  linkcolor=Maroon,
  filecolor=Maroon,
  citecolor=Blue,
  urlcolor=blue,
  pdfcreator={LaTeX via pandoc}}
\urlstyle{same} % disable monospaced font for URLs
\usepackage[margin=1in]{geometry}
\usepackage{color}
\usepackage{fancyvrb}
\newcommand{\VerbBar}{|}
\newcommand{\VERB}{\Verb[commandchars=\\\{\}]}
\DefineVerbatimEnvironment{Highlighting}{Verbatim}{commandchars=\\\{\}}
% Add ',fontsize=\small' for more characters per line
\usepackage{framed}
\definecolor{shadecolor}{RGB}{248,248,248}
\newenvironment{Shaded}{\begin{snugshade}}{\end{snugshade}}
\newcommand{\AlertTok}[1]{\textcolor[rgb]{0.94,0.16,0.16}{#1}}
\newcommand{\AnnotationTok}[1]{\textcolor[rgb]{0.56,0.35,0.01}{\textbf{\textit{#1}}}}
\newcommand{\AttributeTok}[1]{\textcolor[rgb]{0.77,0.63,0.00}{#1}}
\newcommand{\BaseNTok}[1]{\textcolor[rgb]{0.00,0.00,0.81}{#1}}
\newcommand{\BuiltInTok}[1]{#1}
\newcommand{\CharTok}[1]{\textcolor[rgb]{0.31,0.60,0.02}{#1}}
\newcommand{\CommentTok}[1]{\textcolor[rgb]{0.56,0.35,0.01}{\textit{#1}}}
\newcommand{\CommentVarTok}[1]{\textcolor[rgb]{0.56,0.35,0.01}{\textbf{\textit{#1}}}}
\newcommand{\ConstantTok}[1]{\textcolor[rgb]{0.00,0.00,0.00}{#1}}
\newcommand{\ControlFlowTok}[1]{\textcolor[rgb]{0.13,0.29,0.53}{\textbf{#1}}}
\newcommand{\DataTypeTok}[1]{\textcolor[rgb]{0.13,0.29,0.53}{#1}}
\newcommand{\DecValTok}[1]{\textcolor[rgb]{0.00,0.00,0.81}{#1}}
\newcommand{\DocumentationTok}[1]{\textcolor[rgb]{0.56,0.35,0.01}{\textbf{\textit{#1}}}}
\newcommand{\ErrorTok}[1]{\textcolor[rgb]{0.64,0.00,0.00}{\textbf{#1}}}
\newcommand{\ExtensionTok}[1]{#1}
\newcommand{\FloatTok}[1]{\textcolor[rgb]{0.00,0.00,0.81}{#1}}
\newcommand{\FunctionTok}[1]{\textcolor[rgb]{0.00,0.00,0.00}{#1}}
\newcommand{\ImportTok}[1]{#1}
\newcommand{\InformationTok}[1]{\textcolor[rgb]{0.56,0.35,0.01}{\textbf{\textit{#1}}}}
\newcommand{\KeywordTok}[1]{\textcolor[rgb]{0.13,0.29,0.53}{\textbf{#1}}}
\newcommand{\NormalTok}[1]{#1}
\newcommand{\OperatorTok}[1]{\textcolor[rgb]{0.81,0.36,0.00}{\textbf{#1}}}
\newcommand{\OtherTok}[1]{\textcolor[rgb]{0.56,0.35,0.01}{#1}}
\newcommand{\PreprocessorTok}[1]{\textcolor[rgb]{0.56,0.35,0.01}{\textit{#1}}}
\newcommand{\RegionMarkerTok}[1]{#1}
\newcommand{\SpecialCharTok}[1]{\textcolor[rgb]{0.00,0.00,0.00}{#1}}
\newcommand{\SpecialStringTok}[1]{\textcolor[rgb]{0.31,0.60,0.02}{#1}}
\newcommand{\StringTok}[1]{\textcolor[rgb]{0.31,0.60,0.02}{#1}}
\newcommand{\VariableTok}[1]{\textcolor[rgb]{0.00,0.00,0.00}{#1}}
\newcommand{\VerbatimStringTok}[1]{\textcolor[rgb]{0.31,0.60,0.02}{#1}}
\newcommand{\WarningTok}[1]{\textcolor[rgb]{0.56,0.35,0.01}{\textbf{\textit{#1}}}}
\usepackage{graphicx,grffile}
\makeatletter
\def\maxwidth{\ifdim\Gin@nat@width>\linewidth\linewidth\else\Gin@nat@width\fi}
\def\maxheight{\ifdim\Gin@nat@height>\textheight\textheight\else\Gin@nat@height\fi}
\makeatother
% Scale images if necessary, so that they will not overflow the page
% margins by default, and it is still possible to overwrite the defaults
% using explicit options in \includegraphics[width, height, ...]{}
\setkeys{Gin}{width=\maxwidth,height=\maxheight,keepaspectratio}
% Set default figure placement to htbp
\makeatletter
\def\fps@figure{htbp}
\makeatother
\setlength{\emergencystretch}{3em} % prevent overfull lines
\providecommand{\tightlist}{%
  \setlength{\itemsep}{0pt}\setlength{\parskip}{0pt}}
\setcounter{secnumdepth}{5}

\title{Getting Started with R, RStudio, \& R Markdown for ENVH 556: Part I}
\author{Lianne Sheppard}
\date{Created for Winter 2021; printed 26 December, 2020}

\begin{document}
\maketitle

{
\hypersetup{linkcolor=}
\setcounter{tocdepth}{3}
\tableofcontents
}
\hypertarget{why-r-markdown-and-reproducible-reports}{%
\section{Why R Markdown and reproducible
reports?}\label{why-r-markdown-and-reproducible-reports}}

All good data analysis should be reproducible. Markdown is a popular
variant of the syntax often used in Wikis, such as Wikipedia. R Markdown
is an extension of Markdown to support the execution and rendering of R
code within the document. It integrates easily with R to allow you to
produce reproducible reports. We will use these tools and reproducible
research practices to support your development in this course.

\hypertarget{credit}{%
\subsection{Credit}\label{credit}}

This document was modified and expanded from one created by Emily Voldal
in fall 2018. It also incorporates input from Brian High.

\textbf{The output (e.g.~\texttt{.html} or \texttt{.pdf}) and source
(\texttt{.Rmd}) documents are optimally used side-by-side so that you
can see both the code and its result. Before you knit this yourself, you
will need to install the packages \texttt{rmarkdown} and \texttt{knitr}.
(These are already installed on UW-supported RStudio servers.)}

\hypertarget{what-is-markdown-and-why-bother}{%
\subsection{What is Markdown, and why
bother?}\label{what-is-markdown-and-why-bother}}

Just like an R script is better than typing code in the console, using R
Markdown is better than using an R script. R Markdown documents allow
you to save your code, the output that corresponds to your code, and a
record of how each calculation and figure was created. Not only will
this help you with the homework for this class, it's good practice for
doing reproducible research.

Although a \texttt{.Rmd} document looks a lot different from an R
script, the basics of R Markdown are straightforward compared to the
rest of R. After you become familiar with the tricks of R Markdown, you
may find it easier and faster to use than an R script.

\hypertarget{getting-started}{%
\section{Getting started}\label{getting-started}}

\hypertarget{log-onto-rstudio-server}{%
\subsection{Log onto RStudio Server}\label{log-onto-rstudio-server}}

In ENVH 556 we will ensure that all applications and labs work on either
of 2 UW \href{https://www.rstudio.com/}{RStudio} servers running on
virtual Linux machines:

\begin{itemize}
\tightlist
\item
  Plasmid, for DEOHS and EPI students:
  \href{https://plasmid.deohs.washington.edu/}{plasmid.deohs.washington.edu}.
\item
  SPH server, for all students:
  \href{https://rstudio.sph.washington.edu/}{rstudio.sph.washington.edu}
\end{itemize}

For either of these UW servers you will need to install and sign into
\href{https://itconnect.uw.edu/connect/uw-networks/about-husky-onnet/}{Husky
OnNet VPN} using your UW NetID.

Another (non-UW) RStudio server option is
\href{https://rstudio.cloud/}{RStudio Cloud}.

\hypertarget{install-applications-locally-on-your-computer}{%
\subsection{Install applications locally on your
computer}\label{install-applications-locally-on-your-computer}}

You are welcome to do your work locally on your laptop, but we will not
be able to provide support for local installations. You will need the
most recent version of RStudio and R. An easy way to do this is to
follow directions on a tutorial program (Swirl) which begins by walking
you through the process of downloading R and RStudio. Here is the link
to the Swirl tutorial: \url{https://swirlstats.com/students.html}. The
first few Swirl lessons cover the ``Basics of R Programming'' which may
help new R users become familiar with R and RStudio.

To work locally, make sure you have installed the \texttt{rmarkdown}
package. For this document, you will also need to install
\texttt{knitr}. If, at any point during this process, R tells you that
you need to install other packages, do so.

\hypertarget{become-familiar-with-r-markdown}{%
\subsection{Become familiar with R
Markdown}\label{become-familiar-with-r-markdown}}

To open a new R Markdown document (\texttt{.Rmd}), select `File', `New
File', `R Markdown'. You will see a window asking you for some
information about your document; R will use this information to fill in
some code in the \texttt{.Rmd} file, which you can change at any time.
For now, leave the default HTML setting and fill in whatever you want
for the title and author. Save this file, just like you would save an R
script.

This is an R Markdown document. The following text is the beginning
explanation in the text that is included in every new R Markdown
documents you open:

\begin{quote}
Markdown is a simple formatting syntax for authoring HTML, PDF, and MS
Word documents. For more details on using R Markdown see
\url{https://rmarkdown.rstudio.com}.
\end{quote}

\begin{quote}
When you click the \textbf{Knit} button a document will be generated
that includes both content as well as the output of any embedded R code
chunks within the document.
\end{quote}

Every time you open a new \texttt{.Rmd} file, you will see this example.
To turn this code into a nice document, press the `knit' button at the
top of the panel. A window will pop up with the knitted document. Every
time you knit a document, in addition to showing you the preview in R,
it will save the knitted document in the same location as the
\texttt{.Rmd} file.

You can run code line-by-line from the \texttt{.Rmd} document as you
build it. However, this runs in a different environment than the one
used by \texttt{knitr}. So you may have different errors in each. It can
be helpful to knit often.

Here are several good R Markdown resources from
\href{https://rmarkdown.rstudio.com/}{RStudio} to help you get started:

\begin{itemize}
\tightlist
\item
  The \href{https://rmarkdown.rstudio.com/lesson-1.html}{Website}
\item
  The
  \href{https://rstudio.com/wp-content/uploads/2015/03/rmarkdown-reference.pdf}{Reference
  Guide}
\item
  The
  \href{https://github.com/rstudio/cheatsheets/raw/master/rmarkdown-2.0.pdf}{Cheat
  Sheet}
\item
  The \href{https://bookdown.org/yihui/rmarkdown/}{Definitive Guide}
\end{itemize}

The post
\href{https://www.r-bloggers.com/2020/07/getting-started-with-r-markdown-guide-and-cheatsheet/}{Getting
Started with R Markdown --- Guide and Cheatsheet} on R-bloggers is also
very helpful.

\hypertarget{anatomy-of-a-.rmd-file}{%
\section{\texorpdfstring{Anatomy of a \texttt{.Rmd}
file}{Anatomy of a .Rmd file}}\label{anatomy-of-a-.rmd-file}}

\hypertarget{yaml-header}{%
\subsection{YAML header}\label{yaml-header}}

The header is enclosed by dashes and is always at the top of the
\texttt{.Rmd} file. By default, it will include a title, author, date,
and what type of file the \texttt{.Rmd} will knit to. You can change the
text of the title, author, and date here any time. These will show up at
the top of your knitted document. You can also change the file type at
any time. For example, if I write \texttt{word\_document} instead of
\texttt{html\_document}, I will get a Word file. However, be aware that
some commands are specific to certain document types, or just show up
differently. It appears that \texttt{html\_document} is the most
flexible and least fussy type of output.

Whenever you open a new \texttt{.Rmd} file and see the example, leave
the header and delete the rest of the example below the last
\texttt{-\/-\/-}.

\hypertarget{text}{%
\subsection{Text}\label{text}}

To put plain text into an R Markdown document, you don't need anything
special. Text that is black in the \texttt{.Rmd} document is plain text
in the knitted document. Blue text in the \texttt{.Rmd} indicates that
it has been formatted in some way by using special characters (for
example, the knitted text may be bold). Part II of this document gives
details on how to format text.

\hypertarget{chunks}{%
\subsection{Chunks}\label{chunks}}

Interspersed in the text are lines of code; these may have a shaded
background in your \texttt{.Rmd} file; these are called `chunks'. Chunks
start with
\texttt{\textasciigrave{}\textasciigrave{}\textasciigrave{}\{r\}} and
end with \texttt{\textasciigrave{}\textasciigrave{}\textasciigrave{}},
each at the beginning of the line. The position of each chunk determines
where its output ends up in the knitted document. We can control what
the code and output of each chunk look like by changing `chunk options'.
If we removed all the text from a \texttt{.Rmd}, the chunks would make
up the complete R script for that analysis. (The code appendix does this
for you automatically; see below.)

Here is an example code chunk to generate some data and calculate its
mean:

\begin{Shaded}
\begin{Highlighting}[]
\CommentTok{#-----example chunk-----------}

\CommentTok{# Code goes here; output appears below.  (Details about the code in this chunk}
\CommentTok{# are in Part II.)}
\KeywordTok{set.seed}\NormalTok{(}\DecValTok{45}\NormalTok{)}
\NormalTok{a <-}\StringTok{ }\KeywordTok{rnorm}\NormalTok{(}\DataTypeTok{mean=}\DecValTok{0}\NormalTok{, }\DataTypeTok{sd=}\DecValTok{2}\NormalTok{, }\DataTypeTok{n=}\DecValTok{20}\NormalTok{)}
\KeywordTok{mean}\NormalTok{(a)}
\end{Highlighting}
\end{Shaded}

\begin{verbatim}
## [1] 0.22
\end{verbatim}

There is more information on chunks in Part II.

\hypertarget{r-markdown-strategies-to-enhance-reproducibility}{%
\section{R Markdown strategies to enhance
reproducibility}\label{r-markdown-strategies-to-enhance-reproducibility}}

\hypertarget{benefits-of-reproducible-reports}{%
\subsection{Benefits of reproducible
reports}\label{benefits-of-reproducible-reports}}

Reproducible reports with embedded data analyses have many benefits,
including:

\begin{itemize}
\item
  Eliminates typos and transcription errors by pulling results directly
  from R
\item
  Automatically updates your results if you change other code (i.e.~if I
  decided to remove one observation from my data set, I wouldn't have to
  re-type all my numbers)
\item
  Creates a record of exactly how you calculated every number (so
  another scientist could easily reproduce your entire analysis, and you
  will never forget how you calculated something)
\end{itemize}

One way to ensure reproducibility is to use in-line code. This
incorporates R results directly within text. To include code output in a
sentence, we use the format of one backtick followed by ``r'', a space,
and then some R code, and ending with a second backtick. For example, we
can write: ``The mean of the data is 0.22.'' See Part II for details on
inline code.

In the remainder of this section, we suggest strategies that support
reproducibility.

\hypertarget{use-projects-and-relative-file-paths-within-them}{%
\subsection{Use projects and relative file paths within
them}\label{use-projects-and-relative-file-paths-within-them}}

An important principle is to keep your scripts, data files, and all
other inputs and outputs within a project folder. Using an RStudio
``project'' makes this easy, decreasing reliance on ``setwd'' commands
in your scripts. You would not refer to files outside of this project
folder, but instead would use subfolders (e.g., ``data'', ``images'',
etc.) to organize your work within your project. Using project folders
also works nicely with version control (Git) as the project folder
becomes your version control ``repository'' for that project.

Let's make a new RStudio project named ``new\_project''. In the GUI
click:

File --\textgreater{} New Project\ldots{}

In the dialogue box that appears click:

New Directory --\textgreater{} New Project

We can make ``new\_project'' a subdirectory of: ``\textasciitilde/Home''
and click ``Create Project''

From here if your create a new \texttt{.Rmd} file it will automatically
be placed within the project's working directory. We will also set up a
new project in the first lab.

\hypertarget{automate-preparation-of-the-working-environment}{%
\subsection{Automate preparation of the working
environment}\label{automate-preparation-of-the-working-environment}}

When you write code in R Markdown it needs to be completely
self-contained - that is, it can't rely on anything you loaded,
imported, or ran outside of the R Markdown document, e.g., in your
RStudio session. An excellent principle is to automate the preparation
of the working environment in your R Markdown document. This includes
setting options and installing software packages.

\hypertarget{set-options}{%
\subsubsection{Set options}\label{set-options}}

The following chunk is an example of setting \texttt{knitr} options.
(Note: We already set \texttt{knitr} options at the beginning of this
document.)

\begin{Shaded}
\begin{Highlighting}[]
\CommentTok{#-----set knitr options-----------}

\KeywordTok{suppressMessages}\NormalTok{(}\KeywordTok{library}\NormalTok{(knitr))}
\NormalTok{opts_chunk}\OperatorTok{$}\KeywordTok{set}\NormalTok{(}\DataTypeTok{tidy=}\OtherTok{FALSE}\NormalTok{, }\DataTypeTok{cache=}\OtherTok{FALSE}\NormalTok{, }\DataTypeTok{echo=}\OtherTok{TRUE}\NormalTok{, }\DataTypeTok{message=}\OtherTok{FALSE}\NormalTok{)}
\end{Highlighting}
\end{Shaded}

You can adjust these to suit your needs at the time of rendering, such
as disabling echo to make a report for someone who might be distracted
by seeing R code. Or enable cache once your script is complete and
working to allow you to more quickly render the script, such as when
knitting a slide presentation just before you delivering a talk.

For more guidance on \texttt{knitr} options, see
\href{https://yihui.name/knitr/options/}{knitr options} or the
\href{https://cran.r-project.org/web/packages/knitr/vignettes/knitr-refcard.pdf}{knitr
Cheat Sheet}.

\hypertarget{install-and-load-packages}{%
\subsubsection{Install and load
packages}\label{install-and-load-packages}}

A key principle is to only load the packages you will need for your
project.\\
To facilitate this, use the \texttt{pacman} package in R instead of
\texttt{install.packages()} and \texttt{library()}. This will allow your
script to automatically install any packages it needs to without forcing
the installation of a package which has already been installed. Here is
an example:

\begin{Shaded}
\begin{Highlighting}[]
\CommentTok{#-----setup pacman-----------}

\CommentTok{# Not evaluated here since done at the beginning of the file}
\CommentTok{# Load pacman into memory, installing as needed}
\NormalTok{my_repo <-}\StringTok{ 'http://cran.r-project.org'}
\ControlFlowTok{if}\NormalTok{ (}\OperatorTok{!}\KeywordTok{require}\NormalTok{(}\StringTok{"pacman"}\NormalTok{)) \{}\KeywordTok{install.packages}\NormalTok{(}\StringTok{"pacman"}\NormalTok{, }\DataTypeTok{repos =}\NormalTok{ my_repo)\}}

\CommentTok{# Load the other packages, installing as needed}
\CommentTok{# Key principle:  Only load the packages you will need}
\NormalTok{pacman}\OperatorTok{::}\KeywordTok{p_load}\NormalTok{(knitr, tidyverse)}
\end{Highlighting}
\end{Shaded}

The first part installs \texttt{pacman} if it is missing, then the
second part installs and loads the other packages as needed. If you do
this at the top of your script for any packages needed later in your
script, it makes it really easy for people to see what packages your
script depends on. This approach will make it much more likely someone,
particularly a new R user, will be able to run your script and reproduce
your results. Many new R users get completely stuck if they run code
that bombs simply because a package has not been installed. And just
putting in \texttt{install.packages()} calls ``just in case'' will
needlessly slow down your script (each and every time it is run) if the
packages have already been installed.

\hypertarget{use-relative-file-paths-not-absolute-file-paths}{%
\subsection{Use relative file paths, not absolute file
paths}\label{use-relative-file-paths-not-absolute-file-paths}}

Best practice is to not include full paths (``C://Project\_1/data/raw'')
to your files, but uses relative paths (``data/raw'') instead, so they
will be more portable -- able to run on someone else's system. Even
better, use \texttt{file.path()} to construct these paths to make them
platform independent, so a person can use, e.g., Windows, macOS, or
Linux to reproduce your results. Windows users take note: paths like
``C:\textbackslash data\textbackslash raw'' or even
``data\textbackslash raw will'' not work on a Mac. Use ``data/raw'', or
better yet, \texttt{file.path("data",\ "raw")} to solve this problem.

\begin{Shaded}
\begin{Highlighting}[]
\CommentTok{#-----file paths---------}

\CommentTok{# Only works on Windows, as other modern operating systems (macOS, Linux) do }
\CommentTok{# not support "drive letters", such as "P:"}
\NormalTok{full_path <-}\StringTok{ 'P:}\CharTok{\textbackslash{}\textbackslash{}}\StringTok{ENVH556'}
\NormalTok{full_path <-}\StringTok{ 'P:/ENVH556'}

\CommentTok{# Will not work on Windows, exposes your username, and won't work for other }
\CommentTok{# users}
\NormalTok{full_path <-}\StringTok{ '/Users/joanna/ENVH556'}

\CommentTok{# Will work on Windows, macOS, and Linux, etc., if the file is in the user's}
\CommentTok{# "home directory"}
\NormalTok{relative_path <-}\StringTok{ '~/ENVH556'}

\CommentTok{# If you are curious about what the "~" expands to, you can use path.expand()}
\NormalTok{full_path_to_home <-}\StringTok{ }\KeywordTok{path.expand}\NormalTok{(}\StringTok{'~'}\NormalTok{)}
\NormalTok{full_path_to_data <-}\StringTok{ }\KeywordTok{path.expand}\NormalTok{(}\StringTok{'~/ENVH556'}\NormalTok{)}

\CommentTok{# This will work on any system if "ENVH556" is one level below the current folder}
\NormalTok{relative_path <-}\StringTok{ 'ENVH556'}

\CommentTok{# This will work if "ENVH556" is at the same level as current folder, where ".."}
\CommentTok{# means "parent folder", since "ENVH556" will share the same parent as your }
\CommentTok{# current "working directory"}
\NormalTok{relative_path <-}\StringTok{ '../ENVH556'}
\end{Highlighting}
\end{Shaded}

\hypertarget{using-the-working-directory}{%
\subsubsection{Using the working
directory}\label{using-the-working-directory}}

Using working directories (folders) means you don't have to use folder
names when accessing files. A working directory is the place that R will
look for files if you don't give a file path. R is always set to a
certain working directory; you can check where that is like this:

\begin{Shaded}
\begin{Highlighting}[]
\CommentTok{#-----getwd---------}

\KeywordTok{getwd}\NormalTok{()}
\end{Highlighting}
\end{Shaded}

You can make your default working directory the location of your
\texttt{.Rmd} file; it's different when you're running code in your
console or in a .R file. If you want to change that default, you can use
\texttt{setwd()}. For example, let's say my RDS file isn't in the same
folder as my \texttt{.Rmd}:

\begin{Shaded}
\begin{Highlighting}[]
\CommentTok{#-----setwd example---------}

\CommentTok{# Here we use a full file path, but it would be better to use a relative one}
\KeywordTok{setwd}\NormalTok{(}\StringTok{"P:/ENVH556"}\NormalTok{)}

\CommentTok{# Here we use a relative file path, where ".." means "one folder up"}
\KeywordTok{setwd}\NormalTok{(}\StringTok{"../ENVH556"}\NormalTok{)}

\CommentTok{# Here we use a relative file path with file.path(), the recommended method}
\CommentTok{# because this method supports muliple computing platforms (Windows, Mac, etc.)}
\KeywordTok{setwd}\NormalTok{(}\KeywordTok{file.path}\NormalTok{(}\StringTok{".."}\NormalTok{, }\StringTok{"ENVH556"}\NormalTok{))}

\CommentTok{# Now we are ready to read the file}
\NormalTok{DEMS <-}\StringTok{ }\KeywordTok{readRDS}\NormalTok{(}\StringTok{"DEMSCombinedPersonal.rds"}\NormalTok{)}

\CommentTok{# You can see that my working directory changed with getwd()}
\KeywordTok{getwd}\NormalTok{()}
\end{Highlighting}
\end{Shaded}

Using working directories would be especially helpful if you needed to
load lots of different data files and you didn't want to type out all
the file paths. If you always put your \texttt{.Rmd} and data files in
the same location, you should be able to use the default working
directory without typing file paths.

One mild word of warning: there are some issues with working directories
in R Markdown. If you are having trouble, you may want to make sure that
\texttt{setwd()} is in the same chunk as your \texttt{readRDS()}
command. There is also a more
\href{http://pbahr.github.io/tips/2016/04/16/fix_rmarkdown_working_directory_issue}{elegant
solution} to this.

\hypertarget{reporting-code-in-the-appendix}{%
\subsection{Reporting code in the
Appendix}\label{reporting-code-in-the-appendix}}

When you turn in assignments, in addition to your well-written answers
(which should not have any code or raw output), you will need to turn in
the actual code you used. The lab template for ENVH 556 provides you
with some standard appendix code to use in every ENVH 556 document. The
chunk options to accomplish this are:

\begin{verbatim}
ref.label=knitr::all_labels(), echo=TRUE, eval=FALSE, include=TRUE
\end{verbatim}

For an example of how to use this, I added a code appendix at the very
end of this document (see last chunk at the end). Note that you will
still want to \ldots{}

\hypertarget{document-your-code}{%
\subsection{Document your code}\label{document-your-code}}

We encourage you to adopt these best practices when comment your code to
improve readability:

\begin{itemize}
\tightlist
\item
  Comment to keep a running commentary on what your code does.
\item
  The level of detail should be enough to clarify but not to enough to
  annoy.
\item
  Insert comments immediately before the line(s) of code to which they
  apply.
\item
  Add spaces and blank lines as needed to separate code and comments.
\item
  Avoid ``side-commenting'', putting comments at the end of a line of
  code.
\item
  Comments should state what the code does, in the form of an
  \href{https://www.wordnik.com/words/imperative}{imperative}.
\end{itemize}

Here is an example of these put into practice:

\begin{Shaded}
\begin{Highlighting}[]
\CommentTok{#-----comment example-----------}

\CommentTok{# Create a vector of temperatures in degrees Celcius}
\NormalTok{temps <-}\StringTok{ }\KeywordTok{c}\NormalTok{(}\DecValTok{21}\NormalTok{, }\DecValTok{22}\NormalTok{, }\DecValTok{20}\NormalTok{, }\DecValTok{19}\NormalTok{, }\DecValTok{19}\NormalTok{, }\DecValTok{19}\NormalTok{, }\DecValTok{22}\NormalTok{, }\DecValTok{19}\NormalTok{)}

\CommentTok{# Calculate the mean temperature and standard deviation}
\KeywordTok{mean}\NormalTok{(temps)}
\KeywordTok{sd}\NormalTok{(temps)}
\end{Highlighting}
\end{Shaded}

\hypertarget{follow-a-consistent-style}{%
\subsection{Follow a consistent style}\label{follow-a-consistent-style}}

Choose a code style and be consistent within your .Rmd. Following
established styles will make your code more readable and easier to
follow - for both you and others. Here are two popular examples:

\begin{itemize}
\tightlist
\item
  The \href{https://style.tidyverse.org/}{Tidyverse Style Guide}
\item
  Google's \href{https://google.github.io/styleguide/Rguide.xml}{R Style
  Guide}
\end{itemize}

\hypertarget{r-markdown-resources}{%
\section{R Markdown resources}\label{r-markdown-resources}}

\hypertarget{tutorials-for-r-markdown}{%
\subsection{Tutorials for R Markdown:}\label{tutorials-for-r-markdown}}

\begin{itemize}
\tightlist
\item
  \href{https://rmarkdown.rstudio.com/lesson-1.html}{The official
  tutorial} - This tutorial has lots of pictures and is well-organized,
  but only covers the basics.
\item
  \href{https://ourcodingclub.github.io/2016/11/24/rmarkdown-1.html}{Getting
  Started, by `John'} - This is really great for someone who has never
  used R Markdown before, and includes tutorials on setting up R
  Markdown and fixing problems. It also has a great table.
\item
  \href{https://bookdown.org/yihui/rmarkdown/}{A book by Yihui Xie} -
  This goes into a lot of detail, and is really well organized and
  clear.
\item
  \href{http://www.stat.cmu.edu/~cshalizi/rmarkdown/}{Using R Markdown
  for Class Reports, by Cosma Shalizi} - This covers many R Markdown
  capabilities in R Markdown. It doesn't provide a lot of detail, but it
  is a good starting point if you're wondering whether you can do
  something in R Markdown.
\end{itemize}

\hypertarget{useful-r-markdown-cheat-sheets-and-reference-guides}{%
\subsection{Useful R Markdown cheat sheets and reference
guides:}\label{useful-r-markdown-cheat-sheets-and-reference-guides}}

\begin{itemize}
\tightlist
\item
  \href{https://www.rstudio.com/wp-content/uploads/2016/03/rmarkdown-cheatsheet-2.0.pdf}{Official
  cheat sheet}
\item
  \href{https://www.rstudio.com/wp-content/uploads/2015/03/rmarkdown-reference.pdf}{Reference
  guide}
\item
  \href{https://www.ethz.ch/content/dam/ethz/special-interest/math/statistics/sfs/Education/Advanced\%20Studies\%20in\%20Applied\%20Statistics/course-material-1719/Datenanalyse/rmarkdown-2.pdf}{Another
  cheat sheet}
\end{itemize}

\hypertarget{r-for-data-science}{%
\subsection{R for Data Science}\label{r-for-data-science}}

The \href{https://r4ds.had.co.nz/}{R for Data Science (R4DS)} book by
Hadley Wickham and Garrett Grolemund is the current modern standard for
using R.

\begin{itemize}
\tightlist
\item
  \href{https://r4ds.had.co.nz/workflow-projects.html}{Chapter 8}
  describes projects and why you should use them\\
\item
  \href{https://r4ds.had.co.nz/communicate-intro.html}{Chapter 26} is
  the introduction to the Communication section with a brief overview of
  the following chapters\\
\item
  \href{https://r4ds.had.co.nz/r-markdown.html}{Chapter 27} introduces R
  Markdown\\
\item
  \href{https://r4ds.had.co.nz/r-markdown-workflow.html}{Chapter 30}
  covers R Markdown workflow
\end{itemize}

\hypertarget{appendix}{%
\section{Appendix}\label{appendix}}

At a minimum, record version numbers of R and your packages with
\texttt{sessionInfo()} at the end of your script and record the output
as an appendix.

\begin{Shaded}
\begin{Highlighting}[]
\CommentTok{#-----print session information-----}

\KeywordTok{sessionInfo}\NormalTok{()}
\end{Highlighting}
\end{Shaded}

\begin{verbatim}
## R version 4.0.3 (2020-10-10)
## Platform: x86_64-apple-darwin17.0 (64-bit)
## Running under: macOS Catalina 10.15.7
## 
## Matrix products: default
## BLAS:   /Library/Frameworks/R.framework/Versions/4.0/Resources/lib/libRblas.dylib
## LAPACK: /Library/Frameworks/R.framework/Versions/4.0/Resources/lib/libRlapack.dylib
## 
## locale:
## [1] en_US.UTF-8/en_US.UTF-8/en_US.UTF-8/C/en_US.UTF-8/en_US.UTF-8
## 
## attached base packages:
## [1] stats     graphics  grDevices utils     datasets  methods   base     
## 
## other attached packages:
## [1] tinytex_0.28  formatR_1.7   ggplot2_3.3.2 dplyr_1.0.2   readr_1.4.0  
## [6] knitr_1.30    pacman_0.5.1 
## 
## loaded via a namespace (and not attached):
##  [1] magrittr_2.0.1   hms_0.5.3        munsell_0.5.0    tidyselect_1.1.0
##  [5] colorspace_2.0-0 R6_2.5.0         rlang_0.4.9      stringr_1.4.0   
##  [9] tools_4.0.3      grid_4.0.3       gtable_0.3.0     xfun_0.19       
## [13] withr_2.3.0      htmltools_0.5.0  ellipsis_0.3.1   yaml_2.2.1      
## [17] digest_0.6.27    tibble_3.0.4     lifecycle_0.2.0  crayon_1.3.4    
## [21] purrr_0.3.4      vctrs_0.3.6      glue_1.4.2       evaluate_0.14   
## [25] rmarkdown_2.6    stringi_1.5.3    compiler_4.0.3   pillar_1.4.7    
## [29] scales_1.1.1     generics_0.1.0   pkgconfig_2.0.3
\end{verbatim}

For ENVH 556, we also want to see all of your code consolidated at the
end of your R Markdown output. The following code will compile all your
code into an appendix code listing. (This next version is for display in
the rendered document and not for execution. It is followed by a working
version that creates the appendix. Note that the chunk header containing
these options should be a single line of code with no line-wrap.)

\begin{verbatim}
 ```{r appendix, ref.label=knitr::all_labels(), echo=TRUE, eval=FALSE, 
 include=TRUE}
 ```
\end{verbatim}

For ENVH 556, if not already in a template, copy and paste the following
chunk into the end of every lab assignment: (This is the version that
executes.)

\begin{Shaded}
\begin{Highlighting}[]
\CommentTok{#-----setup------}

\CommentTok{# Set knitr options:}
\NormalTok{knitr}\OperatorTok{::}\NormalTok{opts_chunk}\OperatorTok{$}\KeywordTok{set}\NormalTok{(}\DataTypeTok{echo =} \OtherTok{TRUE}\NormalTok{, }\DataTypeTok{warning =} \OtherTok{FALSE}\NormalTok{, }\DataTypeTok{messages =} \OtherTok{FALSE}\NormalTok{, }
                      \DataTypeTok{tidy.opts=}\KeywordTok{list}\NormalTok{(}\DataTypeTok{width.cutoff =} \DecValTok{80}\NormalTok{, }\DataTypeTok{blank =} \OtherTok{TRUE}\NormalTok{) )}

\CommentTok{# Set R option: show only 2 digits when displaying}
\KeywordTok{options}\NormalTok{(}\DataTypeTok{digits =} \DecValTok{2}\NormalTok{)}

\CommentTok{# Clear workspace of all objects and unload all extra (non-base) packages}
\KeywordTok{rm}\NormalTok{(}\DataTypeTok{list =} \KeywordTok{ls}\NormalTok{(}\DataTypeTok{all =} \OtherTok{TRUE}\NormalTok{))}
\ControlFlowTok{if}\NormalTok{ (}\OperatorTok{!}\KeywordTok{is.null}\NormalTok{(}\KeywordTok{sessionInfo}\NormalTok{()}\OperatorTok{$}\NormalTok{otherPkgs)) \{}
\NormalTok{    res <-}\StringTok{ }\KeywordTok{suppressWarnings}\NormalTok{(}
        \KeywordTok{lapply}\NormalTok{(}\KeywordTok{paste}\NormalTok{(}\StringTok{'package:'}\NormalTok{, }\KeywordTok{names}\NormalTok{(}\KeywordTok{sessionInfo}\NormalTok{()}\OperatorTok{$}\NormalTok{otherPkgs), }\DataTypeTok{sep=}\StringTok{""}\NormalTok{),}
\NormalTok{               detach, }\DataTypeTok{character.only=}\OtherTok{TRUE}\NormalTok{, }\DataTypeTok{unload=}\OtherTok{TRUE}\NormalTok{, }\DataTypeTok{force=}\OtherTok{TRUE}\NormalTok{))}
\NormalTok{\}}

\CommentTok{# Load key packages using pacman (see below for explanation)}
\NormalTok{my_repo <-}\StringTok{ 'http://cran.r-project.org'}
\ControlFlowTok{if}\NormalTok{ (}\OperatorTok{!}\KeywordTok{require}\NormalTok{(}\StringTok{"pacman"}\NormalTok{)) \{}\KeywordTok{install.packages}\NormalTok{(}\StringTok{"pacman"}\NormalTok{, }\DataTypeTok{repos =}\NormalTok{ my_repo)\}}

\CommentTok{# Load the other packages, installing as needed}
\CommentTok{# Key principle:  Only load the packages you will need}
\NormalTok{pacman}\OperatorTok{::}\KeywordTok{p_load}\NormalTok{(knitr, readr, dplyr, ggplot2, formatR,tinytex)}

\CommentTok{#-----example chunk-----------}

\CommentTok{# Code goes here; output appears below.  (Details about the code in this chunk}
\CommentTok{# are in Part II.)}
\KeywordTok{set.seed}\NormalTok{(}\DecValTok{45}\NormalTok{)}
\NormalTok{a <-}\StringTok{ }\KeywordTok{rnorm}\NormalTok{(}\DataTypeTok{mean=}\DecValTok{0}\NormalTok{, }\DataTypeTok{sd=}\DecValTok{2}\NormalTok{, }\DataTypeTok{n=}\DecValTok{20}\NormalTok{)}
\KeywordTok{mean}\NormalTok{(a)}

\CommentTok{#-----set knitr options-----------}

\KeywordTok{suppressMessages}\NormalTok{(}\KeywordTok{library}\NormalTok{(knitr))}
\NormalTok{opts_chunk}\OperatorTok{$}\KeywordTok{set}\NormalTok{(}\DataTypeTok{tidy=}\OtherTok{FALSE}\NormalTok{, }\DataTypeTok{cache=}\OtherTok{FALSE}\NormalTok{, }\DataTypeTok{echo=}\OtherTok{TRUE}\NormalTok{, }\DataTypeTok{message=}\OtherTok{FALSE}\NormalTok{)}

\CommentTok{#-----setup pacman-----------}

\CommentTok{# Not evaluated here since done at the beginning of the file}
\CommentTok{# Load pacman into memory, installing as needed}
\NormalTok{my_repo <-}\StringTok{ 'http://cran.r-project.org'}
\ControlFlowTok{if}\NormalTok{ (}\OperatorTok{!}\KeywordTok{require}\NormalTok{(}\StringTok{"pacman"}\NormalTok{)) \{}\KeywordTok{install.packages}\NormalTok{(}\StringTok{"pacman"}\NormalTok{, }\DataTypeTok{repos =}\NormalTok{ my_repo)\}}

\CommentTok{# Load the other packages, installing as needed}
\CommentTok{# Key principle:  Only load the packages you will need}
\NormalTok{pacman}\OperatorTok{::}\KeywordTok{p_load}\NormalTok{(knitr, tidyverse)}

\CommentTok{#-----file paths---------}

\CommentTok{# Only works on Windows, as other modern operating systems (macOS, Linux) do }
\CommentTok{# not support "drive letters", such as "P:"}
\NormalTok{full_path <-}\StringTok{ 'P:}\CharTok{\textbackslash{}\textbackslash{}}\StringTok{ENVH556'}
\NormalTok{full_path <-}\StringTok{ 'P:/ENVH556'}

\CommentTok{# Will not work on Windows, exposes your username, and won't work for other }
\CommentTok{# users}
\NormalTok{full_path <-}\StringTok{ '/Users/joanna/ENVH556'}

\CommentTok{# Will work on Windows, macOS, and Linux, etc., if the file is in the user's}
\CommentTok{# "home directory"}
\NormalTok{relative_path <-}\StringTok{ '~/ENVH556'}

\CommentTok{# If you are curious about what the "~" expands to, you can use path.expand()}
\NormalTok{full_path_to_home <-}\StringTok{ }\KeywordTok{path.expand}\NormalTok{(}\StringTok{'~'}\NormalTok{)}
\NormalTok{full_path_to_data <-}\StringTok{ }\KeywordTok{path.expand}\NormalTok{(}\StringTok{'~/ENVH556'}\NormalTok{)}

\CommentTok{# This will work on any system if "ENVH556" is one level below the current folder}
\NormalTok{relative_path <-}\StringTok{ 'ENVH556'}

\CommentTok{# This will work if "ENVH556" is at the same level as current folder, where ".."}
\CommentTok{# means "parent folder", since "ENVH556" will share the same parent as your }
\CommentTok{# current "working directory"}
\NormalTok{relative_path <-}\StringTok{ '../ENVH556'}

\CommentTok{#-----getwd---------}

\KeywordTok{getwd}\NormalTok{()}

\CommentTok{#-----setwd example---------}

\CommentTok{# Here we use a full file path, but it would be better to use a relative one}
\KeywordTok{setwd}\NormalTok{(}\StringTok{"P:/ENVH556"}\NormalTok{)}

\CommentTok{# Here we use a relative file path, where ".." means "one folder up"}
\KeywordTok{setwd}\NormalTok{(}\StringTok{"../ENVH556"}\NormalTok{)}

\CommentTok{# Here we use a relative file path with file.path(), the recommended method}
\CommentTok{# because this method supports muliple computing platforms (Windows, Mac, etc.)}
\KeywordTok{setwd}\NormalTok{(}\KeywordTok{file.path}\NormalTok{(}\StringTok{".."}\NormalTok{, }\StringTok{"ENVH556"}\NormalTok{))}

\CommentTok{# Now we are ready to read the file}
\NormalTok{DEMS <-}\StringTok{ }\KeywordTok{readRDS}\NormalTok{(}\StringTok{"DEMSCombinedPersonal.rds"}\NormalTok{)}

\CommentTok{# You can see that my working directory changed with getwd()}
\KeywordTok{getwd}\NormalTok{()}

\CommentTok{#-----comment example-----------}

\CommentTok{# Create a vector of temperatures in degrees Celcius}
\NormalTok{temps <-}\StringTok{ }\KeywordTok{c}\NormalTok{(}\DecValTok{21}\NormalTok{, }\DecValTok{22}\NormalTok{, }\DecValTok{20}\NormalTok{, }\DecValTok{19}\NormalTok{, }\DecValTok{19}\NormalTok{, }\DecValTok{19}\NormalTok{, }\DecValTok{22}\NormalTok{, }\DecValTok{19}\NormalTok{)}

\CommentTok{# Calculate the mean temperature and standard deviation}
\KeywordTok{mean}\NormalTok{(temps)}
\KeywordTok{sd}\NormalTok{(temps)}

\CommentTok{#-----print session information-----}

\KeywordTok{sessionInfo}\NormalTok{()}

\CommentTok{#-----appendix-----}
\end{Highlighting}
\end{Shaded}

\end{document}
